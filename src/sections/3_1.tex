% Copyright 2021 Joel Feldman, Andrew Rechnitzer and Elyse Yeager, except where noted.
% This work is licensed under a Creative Commons Attribution-NonCommercial-ShareAlike 4.0 International License.
% https://creativecommons.org/licenses/by-nc-sa/4.0/

\section*{3.1: Velocity and Acceleration}

\begin{frame}{Table of Contents}

\mapofcontentsC{\ca}
\end{frame}

%----------------------------------------------------------------------------------------
%----------------------------------------------------------------------------------------
%----------------------------------------------------------------------------------------
%----------------------------------------------------------------------------------------


%----------------------------------------------------------------------------------------

%----------------------------------------------------------------------------------------
\begin{frame}<1-2>[t]The position of a unicyclist along a tightrope is given by
\[s(t)=t^3-3t^2-9t+10\]
\only<1>{where $s(t)$ gives the distance in meters to the right of the middle of the tightrope, and $t$ is measured in seconds, $-2 \leq t \leq 4$.\\[1em]

Describe the unicyclist's motion: when they are moving right or left; when they are moving fastest and slowest; and how far to the right or left of centre they travel.}

\only<1>{\MoreSpace}
\QuestionBar{1}{5}\AnswerYes
\unote{Example~\eref{text}{eg:velaccelA}}
\end{frame}

%----------------------------------------------------------------------------------------

\begin{frame}<handout:0>[t]
\note<1>{Negative time indices are intentional: students often think these are not possible. Address it if someone asks.}
\AnswerBar{1}{5}
\color{answercolor}\small
\only<1>{The velocity of the unicyclist is given by
\[s'(t)=3t^2-6t-9 = 3(t-3)(t+1)\]
Let's decide where this is positive and negative. It's a parabola pointing up, with zeroes at $t=-1$ and $t=3$. 
\begin{itemize}
\item\color{answercolor}
So, $s'(t)$ is positive when $-2 \leq t <-1$
(so that $t-3< 0$ and $t+1< 0$) and $3<t \leq 4$ (so that $t-3>0$ and $t+1>0$), so these are the times when the unicyclist is moving right. 
\item\color{answercolor}
They are moving left when $-1<t<3$ (so that $t-3<0$ and $t+1>0$).
\end{itemize}
\vfill

The fastest leftward speed corresponds to the minimum of $s'(t)$. This occurs at the ``bottom" of the parabola; to find where this bottom is, we can either remember that parabolas are symmetric (so it occurs halfway between -1 and 3) or we can notice that the minimum occurs when $s'(t)$ is not decreasing any more, but not yet increasing: when $s''(t)=0$. Since $s''(t)=6t-6$, this happens at $t=1$, and at $t=1$, $s'(t)=-12$, so the unicyclist's fastest leftward speed is 12 m/s.

 Their fastest rightward speed will happen at the time farthest from $t=1$, the bottom of the parabola. Their rightward speed is 15m/s when $t=-2$ and $t=4$, and this is their fastest rightward speed.
\vfill}

\only<2>{They are moving the slowest when they switch from left to right motion; at these times, their instantaneous rate of change is zero, and these occur at $t=-1$ and $t=3$.\vfill

It remains to determine how far left and right the unicyclist travels.
\begin{itemize}
\item\color{answercolor}
 $s(-2)=8$, so they start 8 meters to the right of centre; 
\item\color{answercolor}
they continue travelling rightward until $t=-1$. $s(-1)=15$, so when they turn, they are 15 meters to the right of centre. 
\item\color{answercolor}
Then they go left until $t=3$. $s(3)=-17$, so they travel leftward until they are 17 meters to the left of centre. 
\item\color{answercolor}
Then they turn again, and $s(4)=-10$, so they end up 10 meters to the left of centre.
\end{itemize}\vfill}

\unote{Example~\eref{text}{eg:velaccelA}}
\end{frame}
%----------------------------------------------------------------------------------------
\begin{frame}<handout:0>
\note<1>{Mention to students we did not need this graph in order to solve this problem. It's just there to illustrate what's going on : what we usually think of as the y-axis now stands for horizontal (not vertical) position.}
\begin{center}\begin{tikzpicture}
\myaxis{t}{2.2}{4.2}{}{1.5}{1.5}
\draw[C1, very thick] plot[domain=-2:4,yscale=0.1,smooth](\x,{\x*\x*\x-3*\x*\x-9*\x+10});
\xcoord{-2}{-2} \xcoord{4}{4}
\draw[xshift=1cm] (-3.5,-4)--(3,-4);
\foreach \t in {-2,...,4}{
	\POWER{\t}{3}{\ta}
	\POWER{\t}{2}{\tt}
	\MULTIPLY{\tt}{3}{\tb}
	\MULTIPLY{\t}{9}{\tc}
	\SUBTRACT{\ta}{\tb}{\tab}
	\SUBTRACT{\tab}{\tc}{\tabc}
	\ADD{\tabc}{10}{\tabcd}
	\DIVIDE{\tabcd}{10}{\y}
	\ADD{\t}{3}{\s}
	\onslide<\s>{\draw (\t,\y)node[vertex]{};
		\draw[xshift=1cm] (\y*2,-4)node[above,inner sep=0]{\includegraphics[width=5mm]{Clipart/unicycle}};
		\index{\includegraphics[height=5mm]{Clipart/unicycle}
		\href{https://thenounproject.com/term/unicycle/859963/}{`Unicycle'} by 
		\href{https://thenounproject.com/leremy/}{Gan Khoon Lay} is licensed under
		 \CCBYthree~ (accessed 6 July 2021)}
		\draw (.75,-5) node{$t=\t$};
		} }
\end{tikzpicture}\end{center}
\unote{Example~\eref{text}{eg:velaccelA}}
\StatusBar{1}{7}
\end{frame}
%----------------------------------------------------------------------------------------

%----------------------------------------------------------------------------------------
\begin{frame}<1-2>[t]
\QuestionBar{2}{5}\AnswerYes
\only<1>{\MoreSpace}
A solution in a beaker is undergoing a chemical reaction, and its temperature (in degrees Celsius) at $t$ seconds from noon is given by
\[T(t) = t^3+3t^2+4t-273\]
\onslide<1>{\begin{itemize}
\item[1.]
When is the reaction increasing the temperature, and when is it decreasing the temperature? 
\item[2.] What is the slowest rate of change of the temperature?
\end{itemize}}
\end{frame}
%----------------------------------------------------------------------------------------
%----------------------------------------------------------------------------------------
\begin{frame}<handout:0>[t]\color{answercolor}

1. The temperature is always increasing. To see that, remember that a positive derivative means an increasing temperature, and a negative derivative means a decreasing temperature.\vfill

$T'(t)=3t^2+6t+4$. If we try to set $T'(t)=0$, using the quadratic formula, we find the roots are $t=\dfrac{-6\pm\sqrt{36-4(3)(4)}}{6}$, which are not real numbers. So, $T'(t)$ is never zero. Since $T'(t)$ is a parabola pointing up, that means it is always positive, so the temperature is always increasing.\vfill

2. To find when it is increasing the slowest, we need to find the minimum value of its rate of change: the minimum value of $T'(t)$. Since $T'(t)$ is a parabola pointing up, its minimum occurs when it's done decreasing but not yet increasing: when the derivative of $T'(t)$ is zero.\vfill

$T''(t)=6t+6$, so its derivative is zero at $t=-1$. Then the temperature is changing at $T'(-1)=3(-1)^2+6(-1)+4=1$ degree per second.
\end{frame}
%----------------------------------------------------------------------------------------
%----------------------------------------------------------------------------------------
\begin{frame}[t]
\QuestionBar{3}{5}\AnswerYes
You roll a magnetic marble across the floor towards a metal fridge, giving it an initial velocity of 50 centimetres per second. The magnet imparts an acceleration on the magnet of 1 meter per second per second. If the magnet hits the fridge after 2 seconds, how far away was it when you rolled it?



\end{frame}
%----------------------------------------------------------------------------------------
%----------------------------------------------------------------------------------------
\begin{frame}<handout:0>[t]
\AnswerBar{3}{5}
\color{answercolor}
We want to know position and velocity, and we know acceleration. Let's call velocity $v(t)$, where $t$ is measured in seconds and we start pushing the marble at $t=0$. Let's call position $s(t)$. We need some frame of reference for $s(t)$, so let's impose an axis so that $s(0):=0$.

\smallskip
Since acceleration is the derivative of velocity, we know $v'(t)=1$. Anything with a constant slope is a straight line, so $v(t)=t+v_0$ for some constant $v_0$. Since we start pushing the ball with velocity $1/2$ m/s, we must have $v(0)=1/2$, so
\[v(t)=t+1/2.\]
Now consider $s(t)$. Note $s'(t)=v(t)=t+1/2$. We can guess what function $s(t)$ gives us such a derivative. The $1/2$ part we can get from $1/2t+s_0$ for some constant $s_0$. We see that $\frac{d}{dt}[\frac{1}{2}t^2]=t$, so $s(t)=\frac{1}{2}t^2+\frac{1}{2}t+s_0$. Since we defined $s(0)=0$, we need $s_0=0$ so that
\[s(t)=\frac{1}{2}t^2+\frac{1}{2}t.\] 
Since the ball hits the fridge after 2 seconds, it moved from $s(0)=0$ to position $s(2)=\frac{1}{2}2^2+\frac{1}{2}(2)=3$. So, the fridge is three meters from the initial position of the magnet.
\end{frame}
%----------------------------------------------------------------------------------------
%----------------------------------------------------------------------------------------
\begin{frame}[t]
\QuestionBar{4}{5}\AnswerYes\MoreSpace
The deceleration of a particular car while braking is 9 $m/s^2$.\\
\textcolor{C3}{1. Suppose the car needs to stop in 30m. How fast can it be going?} \\(Give your answer in kph.)

\vfill

\textcolor{C1}{2. Suppose the car needs to stop in 50m. How fast can it be going?}\\ (Give your answer in kph.)

\unote{Example~\eref{text}{eg:braking}}
\end{frame}
%----------------------------------------------------------------------------------------
\begin{frame}<handout:0>[t]
\only<1>{\QuestionBar{4}{5}\AnswerYes}
\only<2-3>{\AnswerBar{4}{5}}
The deceleration of a particular car while braking is 9 $m/s^2$.\\
\textcolor{C3}{1. Suppose the car needs to stop in 30m. How fast can it be going?}

\color{answercolor}
\only<2>{\vfill Suppose the car is traveling at $v_0$ kph, and brakes at $t=0$. Then 
\[v(t)=v_0-9t\]
so it stops at $t_{stop}=\frac{v_0}{9}$. We need an expression for how far it has traveled. Let $s(t)$ be its position at $t$ seconds after braking, with $s(0)=0$. Recall $s'(t)=v(t)=v_0-9t$. So, by guessing, 
\[s(t)=v_0t-\frac{9}{2}t^2.\]
Then when the car stops, it has traveled $s(t_{stop})$ meters.
\[s(t_{stop}) = s\left(\frac{v_0}{9}\right)=v_0\left(\frac{v_0}{9}\right)-\frac{9}{2}\left( \frac{v_0}{9}\right)^2=\frac{v_0^2}{18}\]}

\only<3>{\vfill So, to stop in 30 m, we solve
\[\frac{v_0^2}{18}=30\]
which tells us the car can only travel at most about 23.238 m/s. We convert to kph: 
\[23.238 \frac{m}{s}\left( \frac{1 \textup{ km}}{1000 m}\right)
\left( \frac{3600 s}{1 \textup{hr}}\right) \approx 84 \textup{kph}\]}
\unote{Example~\eref{text}{eg:braking}}
\end{frame}
%----------------------------------------------------------------------------------------
\begin{frame}<handout:0>[t]
\only<1>{\QuestionBar{4}{5}\AnswerYes}
\only<2>{\AnswerBar{4}{5}}
\textcolor{C1}{2. Suppose the car needs to stop in 50m. How fast can it be going?}\vfill

\color{answercolor}
\onslide<2>{
To stop in 50 m, we solve
\[\frac{v_0^2}{18}=50\]
which yields $v_0 =30m/s$, or
\[30 \frac{m}{s}\left( \frac{1 \textup{ km}}{1000 m}\right)
\left( \frac{3600 s}{1 \textup{hr}}\right) =108 \textup{kph}\]}
\unote{Example~\eref{text}{eg:braking}}
\end{frame}
%----------------------------------------------------------------------------------------
%----------------------------------------------------------------------------------------
%----------------------------------------------------------------------------------------
\begin{frame}<1-2>[t]
\QuestionBar{5}{5}\AnswerYes
\only<1-2>{\MoreSpace}
Suppose your brakes decelerate your car at a constant rate. That is, $d$ meters per second per second, for some constant $d$.

\only<2>{
\textcolor{C3}{ Is it true that if you double your speed, you double your stopping time?}

}

\vfill
\end{frame}
%----------------------------------------------------------------------------------------
%----------------------------------------------------------------------------------------
\begin{frame}<handout:0>[t]
\AnswerBar{5}{5}
\textcolor{C1}{ Is is true that if you double your speed, you double your stopping distance?}

\color{answercolor}You double your stopping time, but \textit{quadruple} your stopping distance.
\vfill
As in the last example, if you brake at time $t=0$ from a speed of $v_0$ m/s, then $v(t)=v_0-dt$ gives your velocity while decelerating. Then the time you stop is when $v(t)=0$, and this occurs at $t_{stop}=\frac{v_0}{d}$.
\vfill
Notice: if we replace $v_0$ with $2v_0$, then $t_{stop}$ doubles. So doubling your speed does indeed double your stopping time.\vfill

Your position while stopping is given by $s(t)=v_0t-\frac{d}{2}t^2$. Your stopping distance is $s(t_{stop})=s\left( \frac{v_0}{d}\right)
=v_0\left( \frac{v_0}{d}\right)-\frac{d}{2}\left( \frac{v_0}{d}\right)^2 =\frac{v_0^2}{2d}$. So if you replace $v_0$ with $2v_0$, your stopping distance goes up by a factor of 4: it quadruples.\vfill
\end{frame}
%----------------------------------------------------------------------------------------
%----------------------------------------------------------------------------------------
