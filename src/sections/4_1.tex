% Copyright 2021 Joel Feldman, Andrew Rechnitzer and Elyse Yeager, except where noted.
% This work is licensed under a Creative Commons Attribution-NonCommercial-ShareAlike 4.0 International License.
% https://creativecommons.org/licenses/by-nc-sa/4.0/


%----------------------------------------------------------------------------------------
%----------------------------------------------------------------------------------------
\section*{4.1: Antiderivatives}
%----------------------------------------------------------------------------------------
%----------------------------------------------------------------------------------------
%----------------------------------------------------------------------------------------

%----------------------------------------------------------------------------------------
%----------------------------------------------------------------------------------------
\begin{frame}
\centering\Large
4.1 Antiderivatives
\end{frame}
%----------------------------------------------------------------------------------------
\begin{frame}[t]
\unote{Definition~\eref{text}{def:antiderivative}}
\AnswerSpace
\only<4>{\AnswerYes}
\begin{block}
{Basic Question}
What function has derivative $f(x)$?
\end{block}
\pause\vfill

If $F'(x)=f(x)$, we call $F(x)$ an \textcolor{M4}{antiderivative} of $f(x)$.\pause\vfill

\begin{block}
{Examples}
$\diff{}{x}[x^2]=2x$, so $x^2$ is an \emph{antiderivative} of $2x$.
\vspace{5mm}

$\diff{}{x}[x^2+5]=2x$, so $x^2+5$ is (also) an \emph{antiderivative} of $2x$.
\end{block}
\pause\vfill

What is the most general antiderivative of $2x$? \\\color{answercolor}
\pause \answer{\fbox{$x^2+c$}, where we understand $c$ as some constant \\(a number not depending on $x$).}
\end{frame}
%----------------------------------------------------------------------------------------

%----------------------------------------------------------------------------------------
\begin{frame}[t]{Antiderivatives}
Find the most general antiderivative for the following equations.
\[f(x)= 17\]
\onslide<2|handout:0>{\textcolor{answercolor}{$17x+c$}}
\vfill
\[f(x)= m \] where $m$ is a constant.

\onslide<2|handout:0>{\textcolor{answercolor}{$mx+c$}}
\vfill
\note<1>{This is a good time to remind students about lines.}
\end{frame}
%----------------------------------------------------------------------------------------
%----------------------------------------------------------------------------------------
\begin{frame}
\StatusBar{1}{7}
\only<1-6>{\AnswerYes}

	\begin{tabular}{lcl}
		differentiation fact && antidifferentiation fact\\ \hline
		$\diff{}{x}[x^2]=2x$ &$\implies$&antideriv of $2x:$\hfill \answer{$ x^2+c$} \\[10pt] \pause
\iftoggle{printsolutions}{		&&antideriv of $x$:\hfill\pause\answer{$\frac{1}{2}x^2+c$}\\[10pt] \pause
		&&\textcolor{W1}{Check: $\diff{}{x}\left[\frac{1}{2}x^2+c \right]=\pause x$}\pause\\[1em]}{}
		$\diff{}{x}[x^3]=3x^2$ &$\implies$&
		\onslide<7-|handout:0>{antideriv of $3x^2$:\hfill $x^3+c$ \\
		&&antideriv of $ x^2$:}\hfill
		\onslide<7-|handout:0>{ $\frac{1}{3}x^3+c$ }
		\\[10pt]
		$\diff{}{x}[x^4]=4x^3$ &$\implies$& \onslide<7-|handout:0>{antideriv of $4x^3$:\hfill $x^4+c$		
 \\
		&&antideriv of $x^3$\hfill\onslide<7-|handout:0>{$\frac14x^4+c}$\\[1em]}
		$\diff{}{x}[x^5]=5x^4$ &$\implies$& \onslide<7-|handout:0>{antideriv of $5x^4$: \hfill$x^5+c$		
 \\
		&&antideriv of $ x^4$:\hfill
		\onslide<7-|handout:0>{ $\frac{1}{5}x^5+c$ }\\[1em]}
	&& antideriv of $ x^n$:\hfill \onslide<7-|handout:0>{ $\frac{1}{n+1}x^{n+1}+c$\\[1em]}
	\onslide<7-|handout:0>{	&& \color{W1}Check: $\diff{}{x}\left[\frac{1}{n+1}x^{n+1}+c \right]=$\pause $x^n$}
		\end{tabular}
\end{frame}
%----------------------------------------------------------------------------------------
\begin{frame}\AnswerNo
\foreach \x in {1,2,3}{\QuestionBar{\x}{5}}
\begin{block}{Power Rule for Antidifferentiation}
	The most general antiderivative of $x^n$ is $\dfrac{1}{n+1}x^{n+1}+c $
	if $n \neq -1$
	\end{block}\vfill

\begin{itemize}
\item $\ds\diff{}{x}\Big[ \hspace{2cm}\Big]=x^5$\\[2em]
\item $\ds\diff{}{x}\Big[ \hspace{2cm}\Big]=x^3$\\[2em]
\item $\ds\diff{}{x}\Big[ \hspace{2cm}\Big]=\frac12x^3$
\end{itemize}
\end{frame}
%----------------------------------------------------------------------------------------
\begin{frame}\AnswerNo
\unote{Example~\eref{text}{eg antidiff poly}}
\foreach \x in {4,5}{\QuestionBar{\x}{5}}

\begin{block}{Power Rule for Antidifferentiation}
	The most general antiderivative of $x^n$ is $\dfrac{1}{n+1}x^{n+1}+c $
	if $n \neq -1$
	\end{block}\vfill

\begin{itemize}
\item $\ds\diff{}{x}\Big[ \hspace{4cm}\Big]=5x^2-15x+3 $\\[2em]
\item $\ds\diff{}{x}\Big[ \hspace{4cm}\Big]=13\left(5x^{14}-3x^{3/7}+52e^x\right)$
\end{itemize}
\note{Now is a good time to remind students that power functions and exponential functions aren't the same}
\end{frame}
%----------------------------------------------------------------------------------------
%----------------------------------------------------------------------------------------
\begin{frame}[t]
\only<1>{\QuestionBar{1}{3}\AnswerYes}
\only<2>{\AnswerBar{1}{3}}
Find the most general antiderivatives.
\[f(x)= \cos x\]
\onslide<2|handout:0>{\textcolor{answercolor}{$\sin x+c$}}
\vfill 

\[f(x)= \sin x \]\onslide<2|handout:0>{\textcolor{answercolor}{$-\cos x+c$}}
\vfill

\[f(x)= \sec^2 x\]\onslide<2|handout:0>{\textcolor{answercolor}{$\tan x+c$}}
\vfill

\[f(x)= \frac{1}{1+x^2} \]\onslide<2|handout:0>{\textcolor{answercolor}{$\arctan x +c$}}
\vfill

\[f(x)=\frac{1}{1+x^2+2x}\]
\onslide<2|handout:0>{\textcolor{answercolor}{$\frac{-1}{x+1}$}}
\vfill
\end{frame}
%----------------------------------------------------------------------------------------
%----------------------------------------------------------------------------------------
\begin{frame}[t]
\only<1>{\QuestionBar{2}{3}\AnswerYes}
\only<2>{\AnswerBar{2}{3}}

Find the most general antiderivatives.
\[f(x)= 17\cos x+x^5\]\onslide<2|handout:0>{\textcolor{answercolor}{$17\sin x +\frac{1}{6}x^6+c$}}
\vfill 

\[f(x) =\frac{23}{5+5x^2} \]\onslide<2|handout:0>{\textcolor{answercolor}{$\frac{23}{5}\arctan x +c$}}
\vfill

\[f(x)=\frac{23}{5+125x^2}\]\onslide<2|handout:0>{\textcolor{answercolor}{$\frac{23}{25}\arctan (5x) +c$}}
\vfill
\note<1>{Not all of these are to actually do -- more to show that there can be trickiness, and that trickiness will be a big part of next semester. Let students work on them in order, only the fastest students will get to the last examples on each slide.}
\end{frame}
%----------------------------------------------------------------------------------------
%----------------------------------------------------------------------------------------
\begin{frame}[t]
\only<1>{\QuestionBar{3}{3}\AnswerYes}
\only<2>{\AnswerBar{3}{3}}

Find the most general antiderivatives.
\[f(x)=\frac{1}{x}, ~x>0\]\onslide<2|handout:0>{\textcolor{answercolor}{$\ln x+c$}}
\vfill 

\[f(x)= 5x^2-32x^5-17\]\onslide<2|handout:0>{\textcolor{answercolor}{$\frac{5}{3}x^3-\frac{16}{3}x^6-17x+c$}}
\vfill

\[f(x)=\csc x \cot x \]\onslide<2|handout:0>{\textcolor{answercolor}{$-\csc x+c$}}
\vfill

\[f(x)=\frac{5}{\sqrt{1-x^2}}+ 17\]\onslide<2|handout:0>{\textcolor{answercolor}{$5\arcsin x + 17x +c$}}
\vfill
\end{frame}
%----------------------------------------------------------------------------------------
%----------------------------------------------------------------------------------------
\begin{frame}[t]{Chose Your Own Adventure}
\only<1>{\QuestionBar{1}{7}\AnswerYes}
\only<2>{\AnswerBar{1}{7}}
\only<3>{\QuestionBar{2}{7}\AnswerYes}
\only<4>{\AnswerBar{2}{7}}
Antiderivative of \textcolor{M3}{$ \sin x \cos x $}:
\begin{itemize}
\item[A.] $\cos x\sin x+c$
\item[B.] $-\cos x\sin x+c$
\item[C.] $\sin^2 x+c$
\alert<2-|handout:0>{\item[D.] $\frac{1}{2}\sin^2 x +c$}
\item[E.] $\frac{1}{2}\cos^2x\sin^2x+c$
\end{itemize}\vfill

\onslide<3->{
In general, antiderivatives of $x^n$ have the form $\frac{1}{n+1}x^{n+1}$. What is the single exception?
\begin{itemize}
\alert<4|handout:0>{\item[A.] $n=-1$}
\item[B.] $n=0$
\item[C.] $n=1$
\item[D.] $n=e$
\item[E.] $n=1/2$
\end{itemize}
}\vfill
\end{frame}
%----------------------------------------------------------------------------------------
%----------------------------------------------------------------------------------------
%----------------------------------------------------------------------------------------
\begin{frame}[t]{All the Adventures are Calculus, Though}
\only<1>{\QuestionBar{3}{7}\AnswerYes}
\only<2>{\AnswerBar{3}{7}}
\only<3>{\QuestionBar{4}{7}\AnswerYes}
\only<4>{\AnswerBar{4}{7}}

Suppose the velocity of a particle at time $t$ is given by $v(t)=t^2+\cos t + 3$. What function gives its position?
\begin{itemize}
\item[A.] $s(t)=2t-\sin t $
\item[B.] $s(t)=2t-\sin t +c$
{\item[C.] $s(t)=t^3+\sin t +3t +c$}
\alert<2-|handout:0>{\item[D.] $s(t)=\frac{1}{3}t^3+\sin t +3t +c$}
\item[E.] $s(t)=\frac{1}{3}t^2-\sin t +3t +c$
\end{itemize}
\vfill
\onslide<3->{
Suppose the velocity of a particle at time $t$ is given by $v(t)=t^2+\cos t + 3$, and its position at time 0 is given by $s(0)=5$. What function gives its position?
\begin{itemize}
\item[A.]  $s(t)=\frac{1}{3}t^3+\sin t +3t $
\alert<4-|handout:0>{\item[B.]  $s(t)=\frac{1}{3}t^3+\sin t +3t +5$}
{\item[C.] $s(t)=\frac{1}{3}t^3+\sin t +3t +c$}
{\item[D.] $s(t)=5t+c$}
\item[E.] $s(t)=5t+5$
\end{itemize}
}
\end{frame}
%----------------------------------------------------------------------------------------
%----------------------------------------------------------------------------------------
\begin{frame}[t]
\only<1>{\QuestionBar{5}{7}\AnswerYes}
\only<2>{\AnswerBar{5}{7}}

Find all functions $f(x)$ with $f(1)=5$ and $f'(x)=e^{3x+5}$.\pause\vfill
\color{answercolor}

\answer{Antiderivative of $ e^{3x+5}$ is $\frac{1}{3}e^{3x+5}+c$. So we only need to solve for $c$. 

\[5=f(1) = \frac{1}{3}e^{3+5}+c\] implies

\[c=5-\frac{e^8}{3}\]

So 
\[f(x)=\frac{1}{3}e^{3x+5}+5-\frac{e^8}{3}\]}
\end{frame}
%----------------------------------------------------------------------------------------
%----------------------------------------------------------------------------------------
\begin{frame}[t]
\unote{Example~\eref{text}{eg_4_1_2}}
\only<1>{\QuestionBar{6}{7}\AnswerYes}
\only<2>{\AnswerBar{6}{7}}

Let $Q(t)$ be the amount of a radioactive isotope in a sample. Suppose the sample is losing $50 e^{-5t}$ mg per second to decay. If $Q(1)=10{e^{-5}}$mg, find the equation for the amount of the isotope at time $t$.\pause\color{answercolor}\vfill

\answer{
We have 
\[\frac{dQ}{dt} = -50e^{-5t}\] 
Note the negative, since our sample is getting smaller. 

\medskip
Then antidifferentiating, we find $Q(t)=10e^{-5t}+c$ for some constant $c$.

\medskip
 Then since $Q(1)=10e^{-5}$, we see $c=0$.
\vfill
\[\boxed{Q(t)=10e^{-5t}}\]\vfill}
\end{frame}
%----------------------------------------------------------------------------------------
%----------------------------------------------------------------------------------------
\begin{frame}[t]
\only<1>{\QuestionBar{7}{7}\AnswerYes}
\only<2>{\AnswerBar{7}{7}}

Suppose $f'(t)=2t+7$. What is $f(10)-f(3)$?\pause\color{answercolor}\vfill

\answer{From antidifferentiation, we have $f(t)=t^2+7t+c$. Then 
\[f(10)-f(3) = [100+70+c] -[9+21 +c]=170-30=140\]}
\vfill
\end{frame}
%----------------------------------------------------------------------------------------


%----------------------------------------------------------------------------------------
%----------------------------------------------------------------------------------------

%----------------------------------------------------------------------------------------