% Copyright 2021 Joel Feldman, Andrew Rechnitzer and Elyse Yeager, except where noted.
% This work is licensed under a Creative Commons Attribution-NonCommercial-ShareAlike 4.0 International License.
% https://creativecommons.org/licenses/by-nc-sa/4.0/


%----------------------------------------------------------------------------------------
%----------------------------------------------------------------------------------------

%----------------------------------------------------------------------------------------
\begin{frame}
\note{In 2015, there were 6 quizzes during the semester. Their questions are compiled here as a short semester review.

You can show the questions one-by-one by scrolling to the Solutions section.}

This file contains questions spanning CLP-1. It should not be taken as a complete review of the course, but rather as a jumping-off point. If you struggle with one question, go back to review its entire section. Sections are noted at the bottom of each page.
\end{frame}
%----------------------------------------------------------------------------------------


%link to solution
%#1: S or L
%#2: number
\newcommand{\slink}[2]{\hyperlink{sol#1#2}{\beamergotobutton{solution~#1#2}}}
%----------------------------------------------------------------------------------------
%----------------------------------------------------------------------------------------
%----------------------------------------------------------------------------------------
\mode<beamer>{
\section{Questions}
%----------------------------------------------------------------------------------------
%----------------------------------------------------------------------------------------
%----------------------------------------------------------------------------------------

\begin{frame}[t]{Short Answer}
\AnswerYes
\begin{enumerate}[S1.]
\item \label{S1} Find all solutions to $\ds x^3 - 3x^2 - x + 3 =0$
\hfill \slink{S}{1}
\vfill
%
\item  \label{S2} Compute the limit\quad $\ds \lim_{x \to 2} \frac{x-2}{x^2-4}$ \hfill \slink{S}{2}
\vfill
%
\item  \label{S3}  Find all values of $c$ such that the following function is \hfill\slink{S}{3}\\
 continuous:
\[f(x)=\left\{\begin{array}{ccc}
8-cx & \text{if} & x\le c\\
x^2 &  \text{if} & x> c
\end{array}\right.\]
Use the definition of continuity to justify your answer.
\vfill
%
\item  \label{S4} Compute \hfill \slink{S}{4}
$$\lim_{x\to -\infty} \frac{3x+5}{\sqrt{x^2+5}-x}$$
%
\end{enumerate}
\end{frame}
%----------------------------------------------------------------------------------------
\begin{frame}[t]{Short Answer}
\begin{enumerate}[S1.]
\setcounter{enumi}{4}

\item   \label{S5}  Find the equation of the tangent line to the graph of \hfill\slink{S}{5}\
$y=\cos(x)$ at
$x=\dfrac{\pi}{4}$.
\vfill
%
\item  \label{S6}  For what values of $x$ does the derivative of\hfill \slink{S}{6}\\
$\dfrac{\sin(x)}{x^2+6x+5}$ exist?\vfill
\item  \label{S7} Find $f'(x)$ if $f(x)= (x^2+1)^{\sin(x)}$. \hfill\slink{S}{7}
\vfill
%
\item  \label{S8} Consider a function of the form $f(x) = A e^{kx}$ where \hfill\slink{S}{8}\\
$A$ and $k$ are constants. 
If $f(0)=3$ and $f(2)=5$, \\find the constants $A$ and $k$.

\vfill
\end{enumerate}
\end{frame}
%----------------------------------------------------------------------------------------
\begin{frame}[t]{Short Answer}
\begin{enumerate}[S1.]
\setcounter{enumi}{8}

%
\item \label{S9}  Consider a function $f(x)$ which has $f'''(x)=\dfrac{x^3}{10-x^2}$.
Show 
that when we approximate $f(1)$ using its second Maclaurin polynomial, the 
absolute error is less than $\frac{1}{50}=0.02$.   \hfill\slink{S}{9}
\vfill
%
\item  \label{S10}  Estimate $\sqrt{35}$ using a linear approximation   \hfill\slink{S}{10}
\vfill
%
\item \label{S11}  Let $f(x)=x^2-2\pi x - \sin(x)$. Show that there exists a real number $c$ such that $f'(c)=0$.
  \hfill\slink{S}{11}
\vfill
%
\item  \label{S12}  Find the intervals where $f(x)=\frac{\sqrt{x}}{x+6}$ is increasing.
  \hfill\slink{S}{12}
\vfill
%
\end{enumerate}
\end{frame}
%----------------------------------------------------------------------------------------

\begin{frame}{Long Answer}
\AnswerYes
\begin{enumerate}[L1.]
\item  \label{L1}
Compute the limit $\ds \lim_{x\to 1} \frac{\sqrt{x+2}-\sqrt{4-x}}{x-1}$. \hfill\slink{L}{1}
\vfill

\item \label{L2}
Show that there exists at least one real number $c$ such that
$2\tan(c)=c+1$.  \hfill\slink{L}{2}
\vfill

\item \label{L3}
Determine whether the derivative of following function exists at
$x=0$
\begin{align*}
f(x) &=\begin{cases}
  2x^3-x^2 & \text{ if }  x\le 0\\
  x^2\sin\left(\dfrac{1}{x}\right) & \text{ if } x> 0
\end{cases}
\end{align*}
You must justify your answer using the definition of a derivative.

 \hfill\slink{L}{3}
\vfill
\end{enumerate}
\end{frame}
%----------------------------------------------------------------------------------------
\begin{frame}{Long Answer}
\AnswerYes
\begin{enumerate}[L1.]
\setcounter{enumi}{3}
\item \label{L4}
If $x^2\cos(y)+2xe^y = 8$, then find $y'$ at the points where $y=0$.
You must justify your answer.
 \hfill\slink{L}{4}
\vfill

\item \label{L5}
Two particles move in the cartesian plane. Particle A travels on the $x$-axis 
starting at  $(10,0)$ and moving towards the origin with a speed of $2$ units 
per second. Particle B travels on the $y$-axis starting at $(0,12)$ and moving 
towards the origin with a speed of $3$ units per second. What is the rate of 
change of the distance between the two particles when particle A reaches the 
point $(4,0)$?  \hfill\slink{L}{5}
\vfill

\item \label{L6}
Find the global maximum and the global minimum for $f(x)=x^3 - 6x^2 + 2$ on the interval $[3,5]$.
 \hfill\slink{L}{6}
\vfill
\end{enumerate}
\end{frame}
%----------------------------------------------------------------------------------------
%----------------------------------------------------------------------------------------
\section{Solutions}
%----------------------------------------------------------------------------------------
\begin{frame}
\huge\centering
Solutions
\end{frame}
}%%%end of section that only shows up in beamer mode
\mode<handout>{\section{Short Answer}}
%----------------------------------------------------------------------------------------
%----------------------------------------------------------------------------------------
\newcounter{questionnumber}
\setcounter{questionnumber}{1}
%----------------------------------------------------------------------------------------
\newcommand{\qbox}[3]{
	\label{sol#1#2}
	\rule{\textwidth}{1pt}\\
	%
	\mode<beamer>{\hfill \hyperlink{#1#2}{\beamerreturnbutton{back to questions}}\\
	#1\ref{#1#2}.}
	%
	\mode<handout>{
	#1\thequestionnumber
	\stepcounter{questionnumber}}
	%
	\quad #3 \\[1em]
	\rule{\textwidth}{1pt}
	}
%1st argument: long or short
%2nd argument: number
%3rd argument: text
%----------------------------------------------------------------------------------------
%----------------------------------------------------------------------------------------

\begin{frame}[t]
\qbox{S}{1}{Find all solutions to $\ds x^3 - 3x^2 - x + 3 =0$}

\onslide<2-|handout:0>{\textcolor{answercolor}{
\begin{align*}
x^3-3x^2-x+3&=x^2(x-3)-(x-3)\\
&=(x^2-1)(x-3)\\
&=(x+1)(x-1)(x-3)
\end{align*}
The solutions are $x=1$, $x=3$, and $x=-1$.}
\vfill
\unote{Factoring functions is a high-school review topic. It comes in especially handy in Section~\eref{text}{sec curve sketch}, Sketching Graphs}}
\end{frame}
%----------------------------------------------------------------------------------------
\begin{frame}[t]
\qbox{S}{2}{ Compute the limit\quad $\ds \lim_{x \to 2} \frac{x-2}{x^2-4}$}

 \vfill
 \onslide<2-|handout:0>{\textcolor{answercolor}{
 \[\dlimx{2}\frac{x-2}{x^2-4}=\dlimx{2}\frac{x-2}{(x-2)(x+2)}=\dlimx{2}\frac{1}{x+2}=\frac{1}{4}\]}
 \vfill
 }
\unote{ Section~\eref{text}{sec_1_4}: Calculating Limits with Limit Laws}
\end{frame}
%----------------------------------------------------------------------------------------
\begin{frame}[t]\footnotesize
\qbox{S}{3}{
 Find all values of $c$ such that the following function is continuous:
\[f(x)=\left\{\begin{array}{ccc}
8-cx & \text{if} & x\le c\\
x^2 &  \text{if} & x> c
\end{array}\right.\]
Use the definition of continuity to justify your answer.
}
\vfill
\onslide<2-|handout:0>{\color{answercolor}\normalsize
When $x \neq c$, $f(x)$ is continuous. The only difficult spot is when $x=c$. 
\begin{itemize}\color{answercolor}
\item $f(c)=8-c^2$
\item $\dlimx{c^-}f(x)=\dlimx{c^-}(8-cx)=8-c^2$
\item $\dlimx{c^+}f(x)=\dlimx{c^+}(x^2)=c^2$
\end{itemize}
Since $f(x)$ is continuous at $c$ only if $f(c)=\dlimx{c}f(x)$, we see the only values of $c$ that make $f$ continuous are those that satisfy $c^2=8-c^2$. That is, $c=\pm2$.
}
\unote{ Section~\eref{text}{sec_1_6}: Continuity}
\end{frame}

%----------------------------------------------------------------------------------------

\begin{frame}[t]\footnotesize
\qbox{S}{4}{
Compute
$$\lim_{x\to -\infty} \frac{3x+5}{\sqrt{x^2+5}-x}$$}
\vfill
\onslide<2-|handout:0>{\color{answercolor}
We start by factoring out $x$ from both the top and the bottom. 
\begin{align*}
\lim_{x \to - \infty} \frac{3x+5}{\sqrt{x^2+5}-x}\left(\frac{1/x}{1/x}\right)&=\lim_{x \to -\infty}\frac{3+\frac5x}{\frac1x\sqrt{x^2+5}-1}
\intertext{Since $x$ is approaching negative infinity, we can assume $x<0$. Then $x=-|x|=-\sqrt{x^2}$. We'll use this form to push the $\frac1x$ into the square root.}
&=\lim_{x \to -\infty}\frac{3+\frac5x}{-\frac1{\sqrt{x^2}}\sqrt{x^2+5}-1}\\
&=\lim_{x \to -\infty}\frac{3+\frac5x}{-\sqrt{1+\frac{5}{x^2}}-1}
=\frac{3+0}{-\sqrt1-1}= -\frac32
\end{align*}
}
\unote{
Section~\eref{text}{sec_1_5}: Limits at Infinity}
\end{frame}

%----------------------------------------------------------------------------------------

\begin{frame}[t]\small
\qbox{S}{5}{
 Find the equation of the tangent line to the graph of $y=\cos(x)$ at
$x=\dfrac{\pi}{4}$.}

\onslide<2-|handout:0>{
\color{answercolor}
\setlength{\abovedisplayskip}{0 pt}
\begin{align*}
f(x)&= \cos x & f\left(\frac\pi4\right)&=\frac{1}{\sqrt 2}\\
f'(x)&=-\sin x & f'\left(\frac\pi4\right)&=-\frac{1}{\sqrt 2}
\end{align*}
Recall the equation of the tangent line to $y=f(x)$ at $x=a$ is $y=f(a)+f'(a)(x-a)$

\[y=\frac1{\sqrt 2}-\frac{1}{\sqrt 2}\left(x-\frac\pi4\right)\]
}
\unote{
 Section~\eref{text}{sec_2_1}: Revisiting Tangent Lines \qquad
Section~\eref{text}{sec diff trig}: Derivatives of Trigonometric Functions}

\end{frame}
%----------------------------------------------------------------------------------------

\begin{frame}[t]
\qbox{S}{6}{
 For what values of $x$ does the derivative of
$\dfrac{\sin(x)}{x^2+6x+5}$ exist? 
}\vfill

\onslide<2-|handout:0>{
\color{answercolor}
First, note that $x^2+6x+5 = (x+1)(x+5)$, % and that both $\sin(-1)$ and $\sin(-5)$ are nonzero. 
so the function does not exist at either $x=-1$ or $x=-5$. For other values of $x$, using the quotient rule, we see
\[
f'(x)=\frac{(x^2+6x+5)(\cos x) - \sin x (2x+6)}{(x^2+6x+5)^2}\]
which exists over the domain of $f(x)$.
\vfill
All together, the derivative exists for all values of $x$ except $-1$ and $-5$.
\vfill}

\unote{Section~\eref{text}{sec_2_6}: Using the Arithmetic of Derivatives}
\end{frame}
%----------------------------------------------------------------------------------------

%----------------------------------------------------------------------------------------
%----------------------------------------------------------------------------------------
%----------------------------------------------------------------------------------------
\begin{frame}[t]\small
\qbox{S}{7}{Find $f'(x)$ if $f(x)= (x^2+1)^{\sin(x)}$.}
\vfill

\onslide<2-|handout:0>{
\color{answercolor}
$f(x)$ is neither an exponential function (with a constant base) nor a power function (with a constant power). When we see a function raised to a function, we differentiate using logarithmic differentiation.
\begin{align*}
f(x)&= (x^2+1)^{\sin(x)}\\
\log f(x)&=\log\left[ (x^2+1)^{\sin(x)}\right] = \sin x \cdot \log(x^2+1)\\
\frac{f'(x)}{f(x)}&=\sin x \cdot\frac{2x}{x^2+1}+\cos x\cdot\log(x^2+1)\\
f'(x)&=f(x)\left[\sin x \cdot\frac{2x}{x^2+1}+\cos x\cdot\log(x^2+1)\right]\\
&=(x^2+1)^{\sin x}\left[\frac{2x\sin x}{x^2+1}+\cos x \cdot \log(x^2+1)\right]
\end{align*}}
\unote{
Section~\eref{text}{sec diff logs}: The Natural Logarithm}
\end{frame}
%----------------------------------------------------------------------------------------

\begin{frame}[t]
\qbox{S}{8}{
Consider a function of the form $f(x) = A e^{kx}$ where $A$ and $k$ are constants. 
If $f(0)=3$ and $f(2)=5$, find the constants $A$ and $k$.}

\onslide<2-|handout:0>{
\setlength{\abovedisplayskip}{0pt}
\textcolor{answercolor}{
\begin{align*}
3=f(0)&=Ae^0 = A\\
5=f(2)&=3e^{2k} \implies \frac53 = e^{2k}\\
\log_e\left(\frac53\right)&=2k\\
k&=\frac{\log_e(5/3)}{2}
\end{align*}
}}

\unote{This is a review of high school material. This type of calculation comes up in Section~\eref{text}{sec:ExpGthDecay}: Exponential Growth and Decay}

\end{frame}
%----------------------------------------------------------------------------------------
%----------------------------------------------------------------------------------------
%----------------------------------------------------------------------------------------
%----------------------------------------------------------------------------------------

\begin{frame}[t]
\qbox{S}{9}{
Consider a function $f(x)$ which has $f'''(x)=\dfrac{x^3}{10-x^2}$.  Show 
that when we approximate $f(1)$ using its second Maclaurin polynomial, the 
absolute error is less than $\frac{1}{50}=0.02$. }
\vfill

\onslide<2-|handout:0>{\color{answercolor}
For some $c$ between 0 and 1:
\begin{align*}
\big| \underbrace{f(1)-T_2(1)}_{\text{error}} \big|&=\left|\frac{f'''(c)}{3!}(1-0)^3\right|=\frac{1}{6}\left|\frac{c^3}{10-c^2}\right|
\intertext{Since $c$ is between 0 and 1, we note $0<c^3<1$ and $9<10-c^2<10$, so:}
\left|f(1)-T_2(1)\right|&<\frac{1}{6}\left|\frac{1}{9}\right|=\frac{1}{54}<\frac{1}{50}
\end{align*}\vfill}

\unote{Subection~\eref{text}{ssec taylor error}: The Error in the Taylor Polynomial Approximations}
\end{frame}

%----------------------------------------------------------------------------------------

\begin{frame}[t]
\qbox{S}{10}{
 Estimate $\sqrt{35}$ using a linear approximation}
\vfill

\onslide<2-|handout:0>{\color{answercolor}
The general form of a linear approximation is \[L(x)=f(a)+f'(a)(x-a)\]

 If $f(x)=\sqrt{x}$ and $a=36$, then $f(a)=6$ and 
$f'(a)=\frac{1}{2\sqrt{a}}=\frac{1}{12}$. So,
\[L(x)=6+\frac{1}{12}(x-36)\]

Then:
$\sqrt{35}=f(35)\approx L(35)=6+\frac{1}{12}(35-36)=6-\frac{1}{12}={\frac{71}{12}}$
}
\unote{Subsection~\eref{text}{ssec_first_approx}: First Approximation --- the Linear approximation}
\end{frame}


%----------------------------------------------------------------------------------------

\begin{frame}[t]
\qbox{S}{11}{
Let $f(x)=x^2-2\pi x - \sin(x)$. Show that there exists a real number $c$ such that $f'(c)=0$.}
\vfill

\onslide<2-|handout:0>{
\color{answercolor}
We note that $f(x)$ is continuous and differentiable over all real numbers. Since $f(0)=f(2\pi)=0$, by Rolle's Theorem (also by the Mean Value Theorem) there exists some $c$ between 0 and $2\pi$ such that $f'(c)=0$.
}
\unote{Section \eref{text}{sec mvt}: The Mean Value Theorem}
\end{frame}
%----------------------------------------------------------------------------------------

\begin{frame}[t]\small
\qbox{S}{12}{
 Find the intervals where $f(x)=\frac{\sqrt{x}}{x+6}$ is increasing.}
 \vfill
 
\onslide<2-|handout:0>{\color{answercolor}
We find where the first derivative is positive.
\begin{align*}
0<f'(x)&=\frac{(x+6)\frac{1}{2\sqrt x}-\sqrt x}{(x+6)^2}&\mbox{multiply by $(x+6)^2$}\\
  0&<(x+6)\frac{1}{2\sqrt x}-\sqrt x&\mbox{multiply by $2\sqrt x$}\\
0&<(x+6)-2x\\
x&<6
\end{align*}
Note, however, that the function's derivative \textit{does not exist} when $x\le 0$. So the interval is $(0,6)$.
}
\unote{
Section \eref{text}{sec mvt}: The Mean Value Theorem \qquad
Section~\eref{text}{sec curve sketch}: Sketching Graphs}
\end{frame}
%----------------------------------------------------------------------------------------
\mode<handout>{\section{Long Answer}\setcounter{questionnumber}{1}}
%----------------------------------------------------------------------------------------
\begin{frame}[t]\small
\qbox{L}{1}{
 Compute the limit $\ds \lim_{x\to 1} \frac{\sqrt{x+2}-\sqrt{4-x}}{x-1}$.}

\onslide<2-|handout:0>{\color{answercolor}
If we try to do the limit naively we get $0/0$, so we simplify. 
 \begin{align*}
\frac{\sqrt{x+2}-\sqrt{4-x}}{x-1}
  &= \frac{\sqrt{x+2}-\sqrt{4-x}}{x-1}
  \cdot \frac{\sqrt{x+2}+\sqrt{4-x}}{\sqrt{x+2}+\sqrt{4-x}}\\
  &= \frac{(x+2)-(4-x)}{(x-1)(\sqrt{x+2}+\sqrt{4-x})} \\
  &= \frac{2x-2}{(x-1)(\sqrt{x+2}+\sqrt{4-x})} \\
  &= \frac{2}{\sqrt{x+2}+\sqrt{4-x}}\\
\lim_{x\to 1} \frac{\sqrt{x+2}-\sqrt{4-x}}{x-1}&=\lim_{x\to 1}  \frac{2}{\sqrt{x+2}+\sqrt{4-x}}
=\frac{2}{\sqrt 3+\sqrt 3} = \frac{1}{\sqrt 3}
\end{align*}
}
\unote{
Section~\eref{text}{sec_1_4}: Calculating Limits with Limit Laws}
\end{frame}


%----------------------------------------------------------------------------------------
\begin{frame}[t]\footnotesize
\qbox{L}{2}{
Show that there exists at least one real number $c$ such that
$2\tan(c)=c+1$.}

\onslide<2-|handout:0>{\color{answercolor}
\begin{itemize}\color{answercolor}
\item $\tan x$ is continuous on the interval $(-\pi/2, \pi/2)$
\item $x+1$ is a polynomial and therefore continuous for all
real numbers
\item So,  $f(x)=2\tan(x)-x-1$ is a continuous function on the interval $(-\pi/2, \pi/2)$.
\item Set $a=0$. Then $a$ is in the interval  $(-\pi/2, \pi/2)$ and
$$f(a)=2\tan(0)-0-1=0-1=-1<0.$$
\item Set $b=\frac{\pi}{4}$.  Then $b$ is in the interval  $(-\pi/2, \pi/2)$ and
$$f(b)=2\tan\left(\frac\pi4\right) -\frac{\pi}{4} - 1=2-\frac\pi4 - 1=1-\frac\pi4=\frac{4-\pi}{4}>0.$$
\item All together: $f(x)$ is continuous on $[0,\pi/4]$, and $f(0)<0$ while
$f(\pi/4)>0$. Then  the Intermediate Value Theorem guarantees the existence of a
real number $c\in (0,\pi/4)$ such that $f(c)=0$.
\end{itemize}
}
\unote{
Section~\eref{text}{sec_1_6}: Continuity}
\end{frame}


%----------------------------------------------------------------------------------------

\begin{frame}[t]
\qbox{L}{3}{
 Determine whether the derivative of following function exists at
$x=0$
\begin{align*}
f(x) &=\begin{cases}
  2x^3-x^2 & \text{ if }  x\le 0\\
  x^2\sin\left(\dfrac{1}{x}\right) & \text{ if } x> 0
\end{cases}
\end{align*}
You must justify your answer using the definition of a derivative.
}

\onslide<2-|handout:0>{\vfill\color{answercolor}
The function is differentiable at $x=0$ if the following limit exists:
$$\lim_{x\to 0}\frac{f(x)-f(0)}{x-0} = \lim_{x\to 0}\frac{f(x)-0}{x}=\lim_{x\to
0} \frac{f(x)}{x}$$
Note that we used the fact that $f(0)=0$ following the definition of the
first branch, which includes the point $x=0$.}
\unote{Section~\eref{text}{sec def deriv}: Definition of the Derivative \qquad Section~\eref{text}{sec_1_4}: Calculating Limits with Limit Laws}
\end{frame}
%--
\begin{frame}<beamer>
\color{answercolor}
We compute left and right limits of $\frac{f(x)}{x}$ as $x$ goes to 0.
\begin{align*}
\lim_{x\to 0^-}\frac{f(x)}{x}&=\lim_{x\to 0^-}\frac{2x^3-x^2}{x}=\lim_{x\to
0^-} 2x^2-x=0\\
\lim_{x\to 0^+}\frac{f(x)}{x}&=\lim_{x\to 0^+}\frac{x^2\sin\left(\frac{1}{x}\right)}{x}=\lim_{x\to
0^+}x\cdot \sin\left(\frac{1}{x}\right)
\end{align*}
Next use the squeeze theorem. Note, first, that 
$-1\le \sin\left(\frac{1}{x}\right)\le 1$, so that
 $ -x\le x\cdot \sin\left(\frac{1}{x}\right)\le x$.
Note, second, that $\lim\limits_{x \to 0 }x = \lim\limits_{x \to 0 }-x = 0$.
So, by the squeeze theorem, 
\begin{align*}
\lim_{x\to 0^+}\frac{f(x)}{x}
&=\lim_{x\to 0^+}x\cdot\sin\left(\frac{1}{x}\right) =0
\end{align*}


Since the left and right limits match (they're both equal to $0$), we conclude
that indeed $f(x)$ is differentiable at $x=0$ (and its derivative at $x=0$ is
actually equal to $0$).\vfill

\only<beamer>{\unote{Section~\eref{text}{sec def deriv}: Definition of the Derivative \qquad Section~\eref{text}{sec_1_4}: Calculating Limits with Limit Laws}}

\end{frame}

%----------------------------------------------------------------------------------------
\begin{frame}[t]
\qbox{L}{4}{If $x^2\cos(y)+2xe^y = 8$, then find $y'$ at the points where $y=0$.\\
You must justify your answer.}

\onslide<2-|handout:0>{
\begin{itemize}\color{answercolor}
 \item First we find the $x$-coordinates where $y=0$. 
\begin{align*}
  x^2\cos(0)+2xe^0 &= 8 \\
  x^2 +2x - 8 &=0\\
  (x+4)(x-2)&=0
\end{align*}
So $x=2,-4$.
\item Now we use implicit differentiation to get $y'$ in terms of $x,y$:
\begin{align*}
  x^2\cos(y)+2xe^y &= 8 & \text{differentiate both sides} \\
  x^2 \cdot (-\sin y) \cdot y' + 2x \cos y + 2xe^y \cdot y' + 2e^y &= 0 
\end{align*}
}
\unote{Section~\eref{text}{sec_2_11}: Implicit Differentiation}
\end{itemize}\end{frame}
%-----------
\begin{frame}<beamer>
\begin{itemize}\color{answercolor}
\item Now set $y=0$ to get
\begin{align*}
  x^2 \cdot (-\sin 0) \cdot y' + 2x \cos 0 + 2xe^0 \cdot y' + 2e^0 &= 0  \\
  0 + 2x + 2xy' + 2 &=0 \\
  2xy'&=-(2x+2)\\
  y' &=  -\frac{1+x}{x}
\end{align*}
\item So at $(x,y)=(2,0)$ we have $y' = -\frac{3}{2}$, 
\item and at $(x,y)=(-4,0)$ we have $y' = -\frac{3}{4}$.
\end{itemize}

\only<beamer>{\unote{Section~\eref{text}{sec_2_11}: Implicit Differentiation}}
\end{frame}
%----------------------------------------------------------------------------------------
\begin{frame}[t]
\qbox{L}{5}{
Two particles move in the cartesian plane. Particle A travels on the $x$-axis 
starting at  $(10,0)$ and moving towards the origin with a speed of $2$ units 
per second. Particle B travels on the $y$-axis starting at $(0,12)$ and moving 
towards the origin with a speed of $3$ units per second. What is the rate of 
change of the distance between the two particles when particle A reaches the 
point $(4,0)$?}

\onslide<2-|handout:0>{\color{answercolor}\vfill
The position of particle $A$ along the $x$ axis starts at $(10,0)$, and 
moves toward the origin at 2 units per second, so its position is given by $\big(x(t),0\big)$ with $x(t)=10-2t$, where $t$ is measured in seconds. Similarly, the position of $B$ along the $y$ axis is given by $\big(0,y(t)\big)$ with $y(t)=12-3t$. The distance $z(t)$ between the two particles satisfies $z(t)^2=x(t)^2+y(t)^2$. 


}
\unote{Section~\eref{text}{sec rrates}: Related Rates}
\end{frame}
%-------
\begin{frame}<beamer>[t]
\color{answercolor}
When $x(t)=4$, we solve $4=10-2t$ for $t$ and find $t=3$, so $y(3)=12-3(3)=3$. Then $z=5$ when $t=3$.

Differentiating implicitly, $z(t)^2=x(t)^2+y(t)^2$ tells us 
\[2z(t)\diff{z}{t}(t)=2x(t)\diff{x}{t}(t)+2y(t)\diff{y}{t}(t)\]
so, when $t=3$,
\[2(5)\diff{z}{t}(3)=2(4)(-2)+2(3)(-3)=-34\]
Then the distance between the two particles is changing at $-\frac{17}{5}$ units per second.

\only<beamer>{\unote{Section~\eref{text}{sec rrates}: Related Rates}}
\end{frame}
%----------------------------------------------------------------------------------------
\begin{frame}[t]
\qbox{L}{6}{Find the global maximum and the global minimum for $f(x)=x^3 - 6x^2 + 2$ on the interval $[3,5]$.}

\onslide<2-|handout:0>{\color{answercolor}\vfill\small
We compute $f'(x)=3x^2 - 12x$. So $f(x)$ has no singular points (i.e. it is differentiable for all $x$), but has two critical points obtained by solving 
\[ f'(x)=3x(x-4)=0\] 
which yields the two critical points $x=0$ and $x=4$.  Only the critical point 
$x=4$ is in the allowed interval $[3,5]$.

In order to compute the global maximum and the global minimum for $f(x)$ on the interval $[3,5]$, we compute the value of $f$ at the allowed critical point and at the end points of the allowed interval.
$$
f(3)=-25,\quad f(4)=-30\quad\text{and}\quad f(5)=-23.$$
So, the global max is $-23$ while the global min is $-30$.
}
\unote{
Section~\eref{text}{sec optimise}: Optimisation}
\end{frame}

