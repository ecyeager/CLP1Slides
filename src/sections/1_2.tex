% Copyright 2021 Joel Feldman, Andrew Rechnitzer and Elyse Yeager, except where noted.
% This work is licensed under a Creative Commons Attribution-NonCommercial-ShareAlike 4.0 International License.
% https://creativecommons.org/licenses/by-nc-sa/4.0/


 %------------------------------------
\section{1.2 Computing Velocity}
 \begin{frame}{Table of Contents}
\mapofcontentsA{\ab}
 \end{frame}
 %--------

%----------------------------------------------------------------------------------------
\newcommand{\bikegraphbase}{%I want to include the same picture with different scales
\myaxis{t}{0}{10.5}{}{0}{4.25}
\draw (-1.5,4) node[left, rotate=90]{km};
\draw[C2, ultra thick, domain=0:4] plot ({\x*\x/1.5876},\x) node[above left]{$y=s(t)$};
\xcoord{0}{8:00}
\xcoord{10}{8:30}
\ycoord{2}{3}
\ycoord{4}{6}
	}
%----------------------------------------------------------------------------------------
\begin{frame}[t]
\begin{tikzpicture}[scale=0.8]
\bikegraphbase
\onslide<2|handout:0>{\draw (2.5,0)--(2.5,-.5) node[below]{8:07};
\draw[dashed] (2.5,0)--(2.5,2)--(0,2);}
\end{tikzpicture}

\note<2>{8:07 is highlighted to emphasize that I was not ravelling at a constant speed: half the distance was covered in only 7 minutes, the other half took 23. 

I've found that the bike example is a good concept check. When students answer the multiple choice question there's usually a lot who get it wrong, so it's an opportunity to fix misconceptions.}
\end{frame}
%----------------------------------------------------------------------------------------
\begin{frame}[t]
\begin{tikzpicture}[scale=0.8]
\bikegraphbase
\onslide<handout:0>{
\onslide<2-6>{\draw (0,0) node[vertex]{};
\draw (10,4) node[vertex]{};}
\onslide<5-6>{\draw[C3, thick] (0,0)--(10,4);}
\onslide<3-6>{\draw[<->, thick, dashed, C3] (10,0.25)--(10,3.75);
\onslide<4-6>{\draw[<->, thick, dashed, C3] (.25,4)--(9.75,4);
\draw[C3] (5,4) node[above]{1/2 hour};}
\draw (10,2) node[right, C3]{6};}}
\end{tikzpicture}
\vfill

It took $\frac12$ hour to bike 6 km. 12 kph represents the:\only<1-5>{\AnswerYes}
\begin{itemize}
\item[A.] secant line to $y=s(t)$ from $t=8:00$ to $t=8:30$
\alert<6|handout:0>{\item[B.] slope of the secant line to $y=s(t)$ from $t=8:00$ to $t=8:30$}
\item[C.] tangent line to $y=s(t)$ at $t=8:30$
\item[D.] slope of the tangent line to $y=s(t)$ at $t=8:30$
\end{itemize}
\StatusBar{1}{6}
\end{frame}
%----------------------------------------------------------------------------------------
\begin{frame}[t]
\begin{tikzpicture}[scale=0.8]
\bikegraphbase
\onslide<handout:0>{
\onslide<3->{\draw[C3, thick] (0,1.8)--(10.5,4.15);}
\onslide<2->{\draw (8.3,0)--(8.3,-.25) node[below]{8:25};
\draw[dashed] (8.3,0)--(8.3,3.6)--(0,3.6);
\draw (8.3,3.6) node[ vertex]{};}}
\end{tikzpicture}
\vfill
At 8:25, the speedometer on my bike reads 5 kph. 5 kph represents the: \only<1-3>{\AnswerYes}
\begin{itemize}
\item[A.] secant line to $y=s(t)$ from $t=8:00$ to  $t=8:25$
\item[B.] slope of the secant line to $y=s(t)$ from $t=8:00$ to $t=8:25$
\item[C.] tangent line to $y=s(t)$ at $t=8:25$
\alert<4-|handout:0>{\item[D.] slope of the tangent line to $y=s(t)$ at $t=8:25$}
\end{itemize}
\StatusBar{1}{4}
\end{frame}
 %--------


%----------------------------------------------------------------------------------------
%----------------------------------------------------------------------------------------
\note{Now we move into using limits for instantaneous rates of change. Students are usually tired of the bike by now so we change the example.

For next slide: in "one way," remind verbally that instantaneous rate of change is slope of tangent line. Use a straight edge to draw the tangent and make use of the graph paper to get a decent approximation.

We use $y$ for the vertical axis instead of $h$ because $h$ is used later for something else}
\begin{frame}
Suppose the distance from the ground $s$ (in meters) of a helium-filled balloon at time $t$ over a 10-second interval is 
given by $s(t)=t^2$. Try to estimate how fast the balloon is rising when $t=5$. 
\pause
\begin{center}
\begin{tikzpicture}[scale=0.5]
\myaxis{t}{0}{10.2}{y}{0}{10.2}
\draw[step=1cm, ultra thin] (0,0) grid (10,10);
   \draw[domain=0:10,smooth,variable=\x,C2, ultra thick] plot ({\x},{\x*\x*.1})
   node[left,inner sep=0,fill=white] {$y=s(t)$};
   
   \foreach \x in {1,...,10}
   	{\xcoord{\x}{\x}
	\MULTIPLY{\x}{10}{\y}
	\ycoord{\x}{\y}}
	\draw[C3] (5,2.5)node[vertex]{};
\onslide<handout:0>{\onslide<2>{\color{C3}\draw (11,8) node[right]{\parbox{2.75cm}{\raggedright One way:\\ Estimate the slope of the tangent line to the curve}};
\color{black}}
\onslide<3->{\color{C3}\draw (11,5) node[right]{\parbox{2.75cm}{\raggedright Another way:\\ Calculate average rate of change for intervals around 5 that get smaller and smaller.}};
\color{black}}
\onslide<3-7>{\draw (4,1.6) node[C3,vertex]{};}
\foreach \t in {3,...,7}{
	\only<\t>{\draw[thick, C3] plot[domain=3:10](\x,{1.6+(\x-4)*(-\t/10+1.7)});
		\draw (13-\t,{(13-\t)*(13-\t)/10)}) node[C3, vertex]{};}}
\onslide<8>{
	\draw[C3, thick] plot[domain=2:9](\x,{\x-2.5}); }
	}
\end{tikzpicture}\end{center}
\StatusBar{3}{8}
\end{frame}
 %--------
%----------------------------------------------------------------------------------------
\begin{frame}[t]
\only<1>{ Let's look for an algebraic way of determining the velocity of the balloon when $t=5$.}
\onslide<handout:0>{

\only<2-3>{Suppose the interval $[5,~~]$ has length $h$. What is the right endpoint of the interval?}
\only<beamer:2|handout:0>{\QuestionBar{1}{4} \AnswerYes}
\only<beamer:3|handout:0>{\AnswerBar{1}{4}}
%
\only<4-5>{Write the equation for the average (vertical) velocity from $t=5$ to $t=5+h$.}
\only<beamer:4|handout:0>{\QuestionBar{2}{4}\AnswerYes}
\only<beamer:5|handout:0>{\AnswerBar{2}{4}}
%
\only<6-8>{What happens to the velocity when $h$ is very, very small?}
\only<beamer|beamer:6>{\MoreSpace  \AnswerYes}
\only<beamer|beamer:7>{\QuestionBar{3}{4} \AnswerYes}


%
\only<9>{What do you think is the slope of the \emph{tangent} line to the graph when $t=5$? \AnswerNo}
\only<beamer|beamer:9>{\QuestionBar{4}{4}}
%
\only<1-6,9->{\begin{tikzpicture}[scale=.8]
\myaxis{t}{0}{6.5}{y}{0}{3.5}
   \draw[scale=1,domain=0:6,smooth,variable=\x,C2, ultra thick] plot ({\x},{\x*\x*.1})
   node[right] {$y=s(t)=t^2$};
      \foreach \x in {4}
   	{\draw (\x,0)--(\x,-.25) node[below] {$5$};}
   \onslide<2->{
   	\xcoord{6}{}
	\draw[decorate, decoration={brace, amplitude=8pt, mirror}] (4,-1)--(6,-1) node[midway, below, yshift=-.3cm]{$h$};}
\onslide<3->{
	\draw (6,-.2) node[below]{\alert<3|handout:0>{$5+h$}};
	\draw (6,3.6) node[vertex]{};}
\draw (4,1.6) node[vertex]{};
\end{tikzpicture}}
}
%
\only<5-8>{\color{answercolor}
\begin{align*}
\text{vel} &=\frac{\Delta \mbox{ height}}{\Delta \mbox{ time}}= \frac{s(5+h)-s(5)}{(5+h)-5} = \frac{(5+h)^2-5^2}{h}\\
\only<beamer|beamer:8>{&=10+h \text{~when~}h \neq 0
\intertext{When $h$ is very small,}
&\approx 10
}
\end{align*}
}
\only<beamer|beamer:8>{\AnswerBar{3}{4}}

\vfill
\end{frame}
 %--------

%--------------------------------------------
\begin{frame}{Our First Limit}
\note<1>{Now we recap what we did:  the informal calculation first, then using it to introduce limit notation}
Average Velocity, $t=5$ to $t=5+h$:
\begin{align*}
\frac{\Delta s}{\Delta t}&=\frac{s(5+h)-s(5)}{h}\\
\onslide<2->{&=\frac{(5+h)^2-5^2}{h}\\}
\onslide<3->{&=10+h \qquad \mbox{when $h \neq 0$}}
\end{align*}
\onslide<4->{When $h$ is very small,}
\onslide<5->{\[\mbox{Vel} \approx 10\]}
\StatusBar{1}{5}
\end{frame}
%------
\begin{frame}{Limit Notation}
We write:
\[\lim_{h \to 0}(10+h)=10\]
\vfill\pause
We say:
``The limit as $h$ goes to 0 of $(10+h)$ is  10."
\vfill\pause
It means:
As $h$ gets extremely close to 0, $(10+h)$ gets extremely close to 10.

\StatusBar{1}{3}
\end{frame}
 %--------

%----------------------------------------------------------------------------------------
%----------------------------------------------------------------------------------------
%----------------------------------------------------------------------------------------
%----------------------------------------------------------------------------------------
%----------------------------------------------------------------------------------------
%----------------------------------------------------------------------------------------
%----------------------------------------------------------------------------------------
