% Copyright 2021 Joel Feldman, Andrew Rechnitzer and Elyse Yeager, except where noted.
% This work is licensed under a Creative Commons Attribution-NonCommercial-ShareAlike 4.0 International License.
% https://creativecommons.org/licenses/by-nc-sa/4.0/


%------------------------------------------------------
\section*{2.3 Interpretations of the Derivative}

 \begin{frame}{Table of Contents}
\mapofcontentsB{\bc}
 \end{frame}
 %--------

%------------------------------------------------------------------
%\subsection*{2.3: Interpreting the Derivative}
%----------------------------------------------------------------------------------------
\begin{frame}[t]
\begin{block}{Interpreting the Derivative}
The derivative of $f(x)$ at $a$, written $f'(a)$, is the instantaneous rate of change of $f(x)$ when $x=a$.
\end{block}\pause

\begin{QuestionSet}
\SetQuestion{Suppose $P(t)$ gives the number of people in the world at $t$ minutes past midnight, January 1, 2012. Suppose further that $P'(0)=156$. How do you interpret $P'(0)=156?$}
\SetAnswer{Suppose $P(t)$ gives the number of people in the world at $t$ minutes past midnight, January 1, 2012. Suppose further that $P'(0)=156$. How do you interpret $P'(0)=156?$\\\vfill

\textcolor{answercolor}{At midnight of January 1, 2012, the world population was increasing at a rate of 156 people each minute}}

%

\SetQuestion{
Suppose $P(n)$ gives the total profit, in dollars, earned by selling $n$ widgets. How do you interpret $P'(100)$?}
\SetAnswer{
Suppose $P(n)$ gives the total profit, in dollars, earned by selling $n$ widgets. How do you interpret $P'(100)$?\\\vfill
\textcolor{answercolor}{How fast your profit is increasing as you sell more widgets, measured in dollars per widget, at the time you sell Widget \#100. So, roughly the profit earned from the sale of the 101st widget.}
}

%

\SetQuestion{Suppose $h(t)$ gives the height of a rocket $t$ seconds after liftoff. What is the interpretation of $h'(t)$?}
\SetAnswer{Suppose $h(t)$ gives the height of a rocket $t$ seconds after liftoff. What is the interpretation of $h'(t)$?\\\vfill
\textcolor{answercolor}{The speed at which the rocket is rising at time $t$.}}

%

\SetQuestion{Suppose $M(t)$ is the number of molecules of a chemical in a test tube $t$ seconds after a reaction starts. Interpret $M'(t)$.}
\SetAnswer{Suppose $M(t)$ is the number of molecules of a chemical in a test tube $t$ seconds after a reaction starts. Interpret $M'(t)$.\\\vfill
\textcolor{answercolor}{The rate (measured in molecules per second) at which the number of molecules of a certain type is changing. Roughly, how many molecules of that type are being added (or taken away, if negative) per second at time $t$.}
}

%

\SetQuestion{Suppose $G(w)$ gives the diameter  in millimetres of steel wire needed to safely support a load of $w$ kg. Suppose further that $G'(100)=0.01$. How do you interpret $G'(100)=0.01?$}
\SetAnswer{Suppose $G(w)$ gives the diameter  in millimetres of steel wire needed to safely support a load of $w$ kg. Suppose further that $G'(100)=0.01$. How do you interpret $G'(100)=0.01?$\\\vfill
\textcolor{answercolor}{When your load is about 100 kg, you need to increase the diameter of your wire by about 0.01 mm for each kg increase in your load.}
}
\end{QuestionSet}
\end{frame}
%------------------------------------------------------------------

\begin{frame}
\index{Natasha Deshpande, Anoosha Kumar, Rohini Ramaswami. (2014). The Effect of National Healthcare Expenditure on Life Expectancy, page 12. \textit{College of Liberal Arts - Ivan Allen College (IAC), School of Economics: Econometric Analysis Undergraduate Research Papers.}  \url{https://smartech.gatech.edu/handle/1853/51648} (accessed July 2021)}
A paper\footnote{Natasha Deshpande, Anoosha Kumar, Rohini Ramaswami, \textit{The Effect of National Healthcare Expenditure on Life Expectancy}, page 12. 
%Originally accessed 2015 or 2016 via \url{https://smartech.gatech.edu/bitstream/handle/1853/51648/The+Effect+of+National+Healthcare+Expenditure+on+Life+Expectancy.pdf}, but in July 2021 link was broken. 
\\ Remark: physician density is measured as number of doctors per 1000 members of the population.} on the impacts of various factors in average life expectancy contains the following:
\begin{quote}\small
The only statistically significant variable in the model is
physician density. The coefficient for this variable 20.67 indicating that a one unit increase in
physician density leads to a 20.67 unit increase in life expectancy. This variable is also statistically
significant at the 1\% level demonstrating that this variable is very strongly and positively correlated
with quality of healthcare received. This denotes that
access to healthcare is very impactful in terms
of increasing the quality of health in the country. 
\end{quote}
\end{frame}
%----------------------------------------------------------------------------------------
\begin{frame}
If $L(p)$ is the average life expectancy in an area with a density $p$ of physicians, write the statement as a derivative: ``a one unit increase in
physician density leads to a 20.67 unit increase in life expectancy."\pause\vfill

\answer{\textcolor{answercolor}{$L'(p) = 20.67$}}
\end{frame}


%------------------------------------------------------------------

\begin{frame}[t]{Equation of the Tangent Line}
The \alert{tangent line} to $f(x)$ at $a$ has slope $f'(a)$ and passes through the point $(a,f(a))$.\vfill

\answer{\only<-3>{\begin{center}
\begin{tikzpicture}[scale=0.75]
\myaxis{x}{0}{10.5}{y}{0}{4}
\draw[C2, ultra thick, domain=0:3.19] plot[samples=100] ({\x^2}, {\x}) node[below]{$y=f(x)$};
\draw[M4, ultra thick, domain=0:10.2] plot (\x, {.25*\x+1});
\draw[M4] (4,2) node[vertex]{};
\draw[M4] (4,2) node[above]{$(a,f(a))$};

\onslide<2->{
\draw[M4] (10,3.5) node[vertex]{};
\draw[M4] (10,3.5) node[above]{$(x,y)$};}
\end{tikzpicture}\end{center}}}

\answer{\onslide<3->{
Slope of the tangent line:
{\color{M4}
\[f'(a) = \frac{\textup{rise}}{\textup{run}}
=\frac{y-f(a)}{x-a}\]}}
\onslide<5->{Rearranging: 
\textcolor{M4}{
$y-f(a) = f'(a)(x-a)$}
 (equation of tangent line)}}

\end{frame}
%----------------------------------------------------------------------------------------
\begin{frame}[t]\AnswerSpace
\only<3>{\AnswerYes}
\begin{block}{Tangent Line Equation -- Theorem~\eref{text}{thm:DIFFtangentLine}}
The tangent line to the function $f(x)$ at point $a$ is:
\[(y-\textcolor{C4}{f(a)})=\textcolor{M4}{f'(a)}(x-\textcolor{M3}{a})\]
\end{block}\pause

\begin{block}{Point-Slope Formula}
In general, a line with slope $\textcolor{M4}{m}$ passing through point $(\textcolor{M3}{x_1},\textcolor{C4}{y_1})$ has the equation:
\[(y-\textcolor{C4}{y_1})=\textcolor{M4}{m}(x-\textcolor{M3}{x_1})\]
\end{block}\pause


Find the equation of the tangent line to the curve $f(x)=\sqrt{x}$ at $x=9$. (Recall $\diff{}{x}\left[\sqrt{x}\right]=\frac{1}{2\sqrt{x}}$).\pause\color{answercolor}

\answer{
\begin{align*}
a&=9,\quad f(a)=3,\quad
f'(a) = \frac1{2\sqrt 9} = \frac16\\
\alert{ y-3}&\alert {=\frac16(x-9)}
\end{align*}}
\end{frame}
%----------------------------------------------------------------------------------------

%----------------------------------------------------------------------------------------
\begin{frame}
\begin{block}{Memorize}
The tangent line to the function $f(x)$ at point $a$ is:
\[(y-f(a))=f'(a)(x-a)\]
\end{block}
\end{frame}
%----------------------------------------------------------------------------------------
%----------------------------------------------------------------------------------------
\begin{frame}[t]\AnswerSpace
\only<1>{\AnswerYes}
\NowYou~
Let $s(t)=3-0.8t^2$. Then $s'(t)=-1.6t$.
Find the equation for the tangent line to the function $s(t)$ when $t=1$.\pause\color{answercolor}
\answer{\begin{align*}
a&=1,\quad s(a)=2.2,\quad s'(a)=-1.6\\
y-2.2&=-1.6(x-1)
\end{align*}}
\end{frame}

%---------------------------------------------------------------------------------------
%------------------------------------------------
